\documentclass[12pt,a4paper]{report}
\usepackage[usenames,dvipsnames,svgnames,table]{xcolor}
\usepackage[hidelinks]{hyperref} 
\usepackage[latin1]{inputenc}
\usepackage[spanish]{babel}
\usepackage{amsmath}
\usepackage{amsfonts}
\usepackage{amssymb}
\usepackage{graphicx}
\usepackage{enumerate}
\usepackage[latin1]{inputenc}
\usepackage[left=2.5cm,right=2.5cm,top=3cm,bottom=3cm]{geometry}
\title{ENTREGABLE 3 DE T\'OPICOS DE LA COMPUTACION I}
\begin{document}
\begin{center} 
\textbf{\Huge  UNIVERSIDAD NACIONAL DE INGENIER\'IA}
\end{center}
\begin{center} 
\textbf{\LARGE FACULTAD DE CIENCIAS}
\end{center}
\hfill \break
\rule[2mm]{160mm}{0.4mm}
\hfill \break
\begin{center} 
\textbf{\huge ARQUEOPER\'U}
\end{center}
\hfill \break
\hfill \break
\rule[2mm]{160mm}{0.4mm}
\hfill \break
\hfill \break
{\textit Autores:} \hspace{7cm} {\textit Profesor:}
\hfill \break
\hfill \break
{Alumno: Carlos Espinoza} \hspace{3,9cm} { Ciro Nu\'nez}
\hfill \break
{Alumno: Carlos Chauca}
\hfill \break
\hfill \break
\hfill \break
\hfill \break
\hfill \break
\hfill \break
\hfill \break
\hfill \break
\hfill \break
\hfill \break
\hfill \break
\hfill \break
\hfill \break
\rule[2mm]{160mm}{0.4mm}
\hfill \break
\hfill \break
\begin{center}
{\textbf{\LARGE Enlace del proyecto:}}
\end{center}
%\hspace{1cm}
\begin{center}
https://github.com/Carlos-Chauca/my-app-ArqueoPeru
\end{center}
\newpage
\begin{center}
\textbf{\'Avance de la interfaz gr\'afica}
\end{center}
En las siguientes im\'agenes se presentar\'a como se ha ido trabajando la Interfaz Gr\'afica del proyecto.
\begin{center} 
\includegraphics[scale=0.3]{images.png}
\end{center}
\begin{center} 
\includegraphics[scale=0.3]{image.png}
\end{center}
\begin{center} 
\includegraphics[scale=0.3]{image1.png}
\end{center}
\hfill \break
\hfill \break
\begin{center} 
\includegraphics[scale=0.3]{image2.png}
\end{center}
\begin{center} 
\includegraphics[scale=0.3]{image3.png}
\end{center}
\hfill \break
\hfill \break
\begin{center} 
\includegraphics[scale=0.3]{image4.png}
\end{center}
\begin{center} 
\includegraphics[scale=0.3]{image5.png}
\end{center}
\hfill \break
\hfill \break

\textbf{\Large \hspace{0.3cm} Especificaciones de usabilidad }
\begin{itemize}
\item El proyecto una vez terminado incluir\'a un login de usuario.
\item El usuario podr\'a buscar informaci\'on del lugar que seleccione o se interese.
\item Podr\'a ver galer\'ia de fotos.
\item Podr\'a ver rese\'as de viajeros, donde encontrar\'a recomendaciones y asimismo incluir sus propias rese\'as de lo que \'el ha experimentado.
\item Podr\'a buscar n\'umero de contactos de gu\'ias tur\'isticos dependiendo de la zona donde se encuentra.
\item El usuario podr\'a consultar cualquier duda que tenga en nuestra secci\'on 'cont\'actenos', donde se incluir\'a un formulario incluyendo una secci\'on para la consulta que desea hacer.
\item El usuario podr\'a acceder a una p\'aagina de evaluaci\'on e indicar que tal le pareci\'o el aplicativo.
\end{itemize}
\newpage
\hfill \break
\textbf{\Large \hspace{0.3cm} Plan de evaluaci\'on: }\hfill \break \hfill \break
El objetivo de la aplicaci\'on es ser una fuente de informaci\'on que permita compartir experiencias entre viajeros, se busca facilitar la toma de decisiones entre los destinos o viajes que planean realizarse.\newline
La aplicaci\'on contar\'a con 3 tipos de usuarios, los an\'onimos, los registrados y los que har\'an el papel de gu\'ias tur\'isticos
\hfill \break
\hfill \break
\textbf{\Large \hspace{0.3cm} M\'etodos de evaluaci\'on: }\hfill \break \hfill \break
Para la evaluaci\'on se considerar\'a una observaci\'on del uso del aplicativo por parte de todos los usuarios, y se tratar\'a de buscar las respuestas a las preguntas hechas anteriormente.
\hfill \break
Para el uso del aplicativo se pondr\'a al usuario en su entorno natural, aqu\'i se les podr\'a asignar un contexto real para as\'i tener una evaluaci\'on significativa.
\hfill \break
Se evaluar\'a que tanto porcentaje de usuarios an\'onimos cambia su estado a usuarios identificados con la app (caso pr\'actico de gu\'ia o turista).
\hfill \break
\hfill \break
El objetivo del aplicativo es aumentar el conocimiento del p\'ublico a las huacas del pa\'is, asi como incentivar el turismo a \'estas compartiendo informaci\'on respecto a materia tur\'istica entre los distintos usuarios de la aplicaci\'on.
\hfill \break
\hfill \break
\textbf{\Large \hspace{0.3cm} Encuesta al usuario: }\hfill \break
\begin{itemize}
\item ?`Qu\'e te pareci\'o la app?
\item ?`Qu\'e te gust\'o?
\item ?`Qu\'e no te gust\'o?
\item ?`Te cost\'o usar el aplicativo?
\item Entre 1 a 5 estrellas, ?`cu\'anto le dar\'ias?
\end{itemize}




\end{document}


