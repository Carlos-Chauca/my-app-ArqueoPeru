\documentclass[12pt,a4paper]{report}
\usepackage[usenames,dvipsnames,svgnames,table]{xcolor}
\usepackage[hidelinks]{hyperref} 
\usepackage[latin1]{inputenc}
\usepackage[spanish]{babel}
\usepackage{amsmath}
\usepackage{amsfonts}
\usepackage{amssymb}
\usepackage{graphicx}
\usepackage{enumerate}
\usepackage[latin1]{inputenc}
\usepackage[left=2.5cm,right=2.5cm,top=3cm,bottom=3cm]{geometry}
\title{LABORATORIO 4 DE ADMINISTRACI\'ON DE REDES}
\begin{document}
\begin{center} 
\textbf{\Huge  UNIVERSIDAD NACIONAL DE INGENIER\'IA}
\end{center}
\begin{center} 
\textbf{\LARGE FACULTAD DE CIENCIAS}
\end{center}
\hfill \break
\rule[2mm]{160mm}{0.4mm}
\hfill \break
\begin{center} 
\textbf{\huge ARQUEOPER\'U}
\end{center}
\hfill \break
\hfill \break
\rule[2mm]{160mm}{0.4mm}
\hfill \break
\hfill \break
{\textit Autores:} \hspace{7cm} {\textit Profesor:}
\hfill \break
\hfill \break
{Alumno: Carlos Espinoza} \hspace{3,9cm} { Ciro Nu\'nez}
\hfill \break
{Alumno: Carlos Chauca}
\hfill \break
\hfill \break
\hfill \break
\hfill \break
\hfill \break
\hfill \break
\hfill \break
\hfill \break
\hfill \break
\hfill \break
\hfill \break
\hfill \break
\hfill \break
\rule[2mm]{160mm}{0.4mm}
\hfill \break
\hfill \break
\begin{center}
{\textbf{\LARGE Enlace del proyecto:}}
\end{center}
%\hspace{1cm}
\newpage
\begin{center}
\textbf{\'INDICE}
\end{center}
\begin{enumerate}[1.]
\item \textbf{Resumen .................................................................................................... 3}
\item \textbf{Descripci\'on del proyecto ........................................................................... 3}
\begin{enumerate}[2.1.]
\item \textbf{Caracter\'isticas de los usuarios ............................................................ 3}
\item \textbf{Intereses de los usuarios ..................................................................... 4}
\item \textbf{Visi\'on de lo que los usuarios intentan conseguir ................................ 4}
\item \textbf{Caracter\'isticas del entorno del \'area ................................................... 4}
\end{enumerate}
\item \textbf{Contexto social .......................................................................................... 4}
\item \textbf{Criterios de usabilidad .............................................................................. 4}
\item \textbf{Prototipo de fidelidad baja ....................................................................... 5}
\item \textbf{Prototipo de fidelidad media (FIGMA) .................................................... 5}
\item \textbf{FIGMA ..................................................................................................... 6}
\end{enumerate}
\newpage
\leftline{\textbf{\Large 1. \hspace{0.3cm} Resumen}}
\hfill \break
Hoy en d\'ia los aplicativos m\'oviles han tomado una gran relevancia en la vida cotidiana de las personas alrededor del mundo, a tal punto que algunas de sus actividades diarias dependen de una app, en este sentido y teniendo \'esto en cuenta se han ido desarrollando m\'as y m\'as aplicativos con el fin de facilitar las actividades de diferentes tipos de usuarios.\newline
\hfill \break
ArqueoPer\'u es un aplicativo m\'ovil que busca orientar a las personas que desconocen acerca de la inmensa diversidad cultural que existe en el pais, esta app se enfoca en orientar a las personas a conocer acerca de las Huacas del Per\'u, lugares con alta informaci\'on y cultura acerca de nuestro pais, antepasados, antiguas civilizaciones, etc; tambi\'en brinda comentarios y recomendaciones de otros usuarios seg\'un lo que experimentan al usar esta app, la aplicaci\'on tambien brinda la posibilidad de contactar a un gu\'ia, etc.
\hfill \break
\hfill \break
\hfill \break
\textbf{\Large 2. \hspace{0.3cm} Descripci\'on del proyecto}
\hfill \break
\begin{enumerate}[2.1.]
\item \textbf{\hspace{0.3cm} Caracter\'isticas de los usuarios}
\hfill \break
\hfill \break
El proyecto tendr\'a los siguientes tipos de usuarios:
\hfill \break
\begin{itemize}
\item Viajeros constantes (turistas): Aquellas personas que suelen venir al pa\'is de visita pero solo conocen Cusco como \'unico lugar para visitar, esta app tendr\'a en cuenta las necesidades y requerimentos que una persona busca seg\'un la necesidad del usuario.
\item Gu\'ias tur\'isticos: Algunas personas residentes cerca a las Huacas del pa\'is, pueden brindar servicios de gu\'ias para los turistas, generando con \'esto algunos ingresos.
\item Aportantes: \'Estas personas son aquellas que ante alguna interrogante dentro de la aplicaci\'on brindan respuestas, las cuales son valoradas y calificadas seg\'un su utilidad.
\item Inversionistas: Aquellos que est\'an interesados en \'este tipo de aplicativos, no intervienen directamente pero son considerados usuarios de la app (aqu\'i se enfoca en aquel p\'ublico relacionado a la tecnolog\'ia).\newline
Lo que se tratar\'a de lograr es que las personas (tanto extranjeras como residentes del pa\'is) conozcan m\'as acerca de la cultura peruana.
\item Estudiantes: Aqu\'i est\'an aquellos estudiantes de instituciones educativas p\'ublicas y / o privadas.
\end{itemize}
\hfill \break
\newpage
\item \textbf{\hspace{0.3cm} Intereses de los usuarios}
\hfill \break
Gracias a ArqueoPeru aquellas personas interesadas en conocer un poco m\'as acerca del Per\'u podr\'an tener acceso a toda la informaci\'on necesaria para conocer sobre la diversidad del pa\'is (Huacas) donde se incluir\'a informaci\'on sobre los lugares arqueol\'ogicos seleccionados por el usuario, \'este tambi\'en incluir\'a una gu\'ia para los visitantes y / o turistas, tambi\'en habra una secci\'on sobre todo tipo de eventos en los lugares arqueol\'ogicos registrados.
\hfill \break
\item \textbf{\hspace{0.3cm} Caracter\'isticas de las tareas realizadas por los usuarios}
\hfill \break
Los usuarios pueden:
\begin{itemize}
\item Ver la descripci\'on del lugar, \'esta app brinda informaci\'on acerca de las Huacas m\'as cercanas respecto a tu posici\'on, su horario de cierre, etc.
\item Interactuar con alg\'una persona que haya visitado previamente el lugar que se est\'a referenciando.
\item Ver rese\'nas de otros usuarios, recomendaciones.
\item Hacer comentarios sobre alg\'un lugar que les interes\'o en particular.
\end{itemize}

\item \textbf{\hspace{0.3cm} Caracter\'isticas del entorno del \'area}
\begin{itemize}
\item Ayudar a aquellas personas que quieren visitar el lugar a escoger el mejor recorrido para tener un mejor viaje.
\item Ayudar a nativos de la zona a generar un ingreso.
\item Eventos: Indicar\'a al usuario acerca de los pr\'oximos eventos del lugar.
\item Promover la cultura peruana.
\end{itemize}

\end{enumerate}

\hfill \break
\hfill \break
\textbf{\Large 3. \hspace{0.3cm} Contexto social}
\hfill \break
\hfill \break
El sistema promover\'a el turismo en el pa\'is y ayudar\'a a los viajeros a disfrutar del viaje al obtener ayuda del lugar donde piensan viajar, podr\'an tambien ver una peque\'na descripci\'on del lugar y hacer filtros de aquellos lugares que deseen visitar, aquellos con el mejor ambiente; tambi\'en podr\'ian solicitar gu\'ias tur\'isticos o leer rese\'nas de viajeros, de este modo su visita se har\'ia mas agradable.
\hfill \break
\hfill \break
\hfill \break
\textbf{\Large 4. \hspace{0.3cm} Criterios de usabilidad }
\begin{itemize}
\item El proyecto una vez terminado incluir\'a un login de usuario.
\item El usuario podr\'a buscar informaci\'on del lugar que seleccione o se interese.
\item Podr\'a ver galer\'ia de fotos.
\item Podr\'a ver rese\'as de viajeros, donde encontrar\'a recomendaciones y asimismo incluir sus propias rese\'as de lo que \'el ha experimentado.
\item Podr\'a buscar n\'umero de contactos de gu\'ias tur\'isticos dependiendo de la zona donde se encuentra.

\end{itemize}


\end{document}


